\section{Task Formulation}
In a simple scenario, the CGA forecasting task can be formulated as a classification problem, mathematically expressed as:
\begin{equation}
p_{\mathrm{derailment}} = f(\mathrm{context})
\label{eq:classification}
\end{equation}
where $f$ is a classification model, \emph{context} represents a static snapshot of the conversation at a specific point in time, and $p_{\mathrm{derailment}}$ is the predicted probability of derailment as output by $f$.

In real-world applications, however, derailment forecasting must occur dynamically, following each utterance in an ongoing conversation. This introduces additional complexities, requiring a temporal dimension to Equation \ref{eq:classification}:
\begin{equation}
p_{\mathrm{event}}(t) = f(\mathrm{context}(t))
\label{eq:forecastingmodel}
\end{equation}

In this formulation, the CGA model produces \emph{multiple} utterance-level forecasts throughout the lifetime of a conversation. However, each conversation is associated with a single ground-truth label, making it unsuitable to treat each forecast as an independent prediction. This necessitates adjustments to both the prediction process and the evaluation methodology.

Under the static classification framework (Equation \ref{eq:classification}), a single prediction corresponds to a specific timestamp in the conversation, referred to as the \textbf{utterance-level} prediction. In contrast, dynamic forecasting requires \emph{aggregating} multiple utterance-level predictions over the course of a conversation to generate a single prediction that can be compared with the ground-truth label.

In this work, we adopt an evaluation approach informed by previous research. Specifically, a conversation is forecasted to derail if any utterance-level prediction exceeds a threshold determined on the development set. This thresholding mechanism aggregates predictions across utterances, producing the final \textbf{conversation-level} forecast. The highest utterance-level forecast for a conversation is used as the conversation-level prediction for evaluation purposes.